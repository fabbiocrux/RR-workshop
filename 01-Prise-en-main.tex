% Options for packages loaded elsewhere
\PassOptionsToPackage{unicode}{hyperref}
\PassOptionsToPackage{hyphens}{url}
%
\documentclass[
]{article}
\usepackage{lmodern}
\usepackage{amsmath}
\usepackage{ifxetex,ifluatex}
\ifnum 0\ifxetex 1\fi\ifluatex 1\fi=0 % if pdftex
  \usepackage[T1]{fontenc}
  \usepackage[utf8]{inputenc}
  \usepackage{textcomp} % provide euro and other symbols
  \usepackage{amssymb}
\else % if luatex or xetex
  \usepackage{unicode-math}
  \defaultfontfeatures{Scale=MatchLowercase}
  \defaultfontfeatures[\rmfamily]{Ligatures=TeX,Scale=1}
\fi
% Use upquote if available, for straight quotes in verbatim environments
\IfFileExists{upquote.sty}{\usepackage{upquote}}{}
\IfFileExists{microtype.sty}{% use microtype if available
  \usepackage[]{microtype}
  \UseMicrotypeSet[protrusion]{basicmath} % disable protrusion for tt fonts
}{}
\makeatletter
\@ifundefined{KOMAClassName}{% if non-KOMA class
  \IfFileExists{parskip.sty}{%
    \usepackage{parskip}
  }{% else
    \setlength{\parindent}{0pt}
    \setlength{\parskip}{6pt plus 2pt minus 1pt}}
}{% if KOMA class
  \KOMAoptions{parskip=half}}
\makeatother
\usepackage{xcolor}
\IfFileExists{xurl.sty}{\usepackage{xurl}}{} % add URL line breaks if available
\IfFileExists{bookmark.sty}{\usepackage{bookmark}}{\usepackage{hyperref}}
\hypersetup{
  pdftitle={Prise en Main},
  pdfauthor={Fabio CRUZ},
  hidelinks,
  pdfcreator={LaTeX via pandoc}}
\urlstyle{same} % disable monospaced font for URLs
\usepackage[margin=1in]{geometry}
\usepackage{color}
\usepackage{fancyvrb}
\newcommand{\VerbBar}{|}
\newcommand{\VERB}{\Verb[commandchars=\\\{\}]}
\DefineVerbatimEnvironment{Highlighting}{Verbatim}{commandchars=\\\{\}}
% Add ',fontsize=\small' for more characters per line
\usepackage{framed}
\definecolor{shadecolor}{RGB}{248,248,248}
\newenvironment{Shaded}{\begin{snugshade}}{\end{snugshade}}
\newcommand{\AlertTok}[1]{\textcolor[rgb]{0.94,0.16,0.16}{#1}}
\newcommand{\AnnotationTok}[1]{\textcolor[rgb]{0.56,0.35,0.01}{\textbf{\textit{#1}}}}
\newcommand{\AttributeTok}[1]{\textcolor[rgb]{0.77,0.63,0.00}{#1}}
\newcommand{\BaseNTok}[1]{\textcolor[rgb]{0.00,0.00,0.81}{#1}}
\newcommand{\BuiltInTok}[1]{#1}
\newcommand{\CharTok}[1]{\textcolor[rgb]{0.31,0.60,0.02}{#1}}
\newcommand{\CommentTok}[1]{\textcolor[rgb]{0.56,0.35,0.01}{\textit{#1}}}
\newcommand{\CommentVarTok}[1]{\textcolor[rgb]{0.56,0.35,0.01}{\textbf{\textit{#1}}}}
\newcommand{\ConstantTok}[1]{\textcolor[rgb]{0.00,0.00,0.00}{#1}}
\newcommand{\ControlFlowTok}[1]{\textcolor[rgb]{0.13,0.29,0.53}{\textbf{#1}}}
\newcommand{\DataTypeTok}[1]{\textcolor[rgb]{0.13,0.29,0.53}{#1}}
\newcommand{\DecValTok}[1]{\textcolor[rgb]{0.00,0.00,0.81}{#1}}
\newcommand{\DocumentationTok}[1]{\textcolor[rgb]{0.56,0.35,0.01}{\textbf{\textit{#1}}}}
\newcommand{\ErrorTok}[1]{\textcolor[rgb]{0.64,0.00,0.00}{\textbf{#1}}}
\newcommand{\ExtensionTok}[1]{#1}
\newcommand{\FloatTok}[1]{\textcolor[rgb]{0.00,0.00,0.81}{#1}}
\newcommand{\FunctionTok}[1]{\textcolor[rgb]{0.00,0.00,0.00}{#1}}
\newcommand{\ImportTok}[1]{#1}
\newcommand{\InformationTok}[1]{\textcolor[rgb]{0.56,0.35,0.01}{\textbf{\textit{#1}}}}
\newcommand{\KeywordTok}[1]{\textcolor[rgb]{0.13,0.29,0.53}{\textbf{#1}}}
\newcommand{\NormalTok}[1]{#1}
\newcommand{\OperatorTok}[1]{\textcolor[rgb]{0.81,0.36,0.00}{\textbf{#1}}}
\newcommand{\OtherTok}[1]{\textcolor[rgb]{0.56,0.35,0.01}{#1}}
\newcommand{\PreprocessorTok}[1]{\textcolor[rgb]{0.56,0.35,0.01}{\textit{#1}}}
\newcommand{\RegionMarkerTok}[1]{#1}
\newcommand{\SpecialCharTok}[1]{\textcolor[rgb]{0.00,0.00,0.00}{#1}}
\newcommand{\SpecialStringTok}[1]{\textcolor[rgb]{0.31,0.60,0.02}{#1}}
\newcommand{\StringTok}[1]{\textcolor[rgb]{0.31,0.60,0.02}{#1}}
\newcommand{\VariableTok}[1]{\textcolor[rgb]{0.00,0.00,0.00}{#1}}
\newcommand{\VerbatimStringTok}[1]{\textcolor[rgb]{0.31,0.60,0.02}{#1}}
\newcommand{\WarningTok}[1]{\textcolor[rgb]{0.56,0.35,0.01}{\textbf{\textit{#1}}}}
\usepackage{graphicx}
\makeatletter
\def\maxwidth{\ifdim\Gin@nat@width>\linewidth\linewidth\else\Gin@nat@width\fi}
\def\maxheight{\ifdim\Gin@nat@height>\textheight\textheight\else\Gin@nat@height\fi}
\makeatother
% Scale images if necessary, so that they will not overflow the page
% margins by default, and it is still possible to overwrite the defaults
% using explicit options in \includegraphics[width, height, ...]{}
\setkeys{Gin}{width=\maxwidth,height=\maxheight,keepaspectratio}
% Set default figure placement to htbp
\makeatletter
\def\fps@figure{htbp}
\makeatother
\setlength{\emergencystretch}{3em} % prevent overfull lines
\providecommand{\tightlist}{%
  \setlength{\itemsep}{0pt}\setlength{\parskip}{0pt}}
\setcounter{secnumdepth}{-\maxdimen} % remove section numbering
\ifluatex
  \usepackage{selnolig}  % disable illegal ligatures
\fi

\title{Prise en Main}
\author{Fabio CRUZ}
\date{}

\begin{document}
\maketitle

{
\setcounter{tocdepth}{3}
\tableofcontents
}
\hypertarget{prise-en-main}{%
\section{Prise en Main}\label{prise-en-main}}

\hypertarget{comme-une-calculatrice}{%
\subsection{Comme une calculatrice:}\label{comme-une-calculatrice}}

\begin{itemize}
\tightlist
\item
  Vous voyez l'opérateur d'assignation \texttt{\textless{}-} :
\end{itemize}

\begin{Shaded}
\begin{Highlighting}[]
\NormalTok{a }\OtherTok{\textless{}{-}} \DecValTok{2}\SpecialCharTok{+}\DecValTok{2}
\NormalTok{b }\OtherTok{\textless{}{-}} \DecValTok{5{-}7}
\NormalTok{c }\OtherTok{\textless{}{-}} \DecValTok{4}\SpecialCharTok{*}\DecValTok{12}
\NormalTok{d }\OtherTok{\textless{}{-}} \DecValTok{10}\SpecialCharTok{/}\DecValTok{3}
\NormalTok{e }\OtherTok{\textless{}{-}} \DecValTok{5}\SpecialCharTok{\^{}}\DecValTok{2}
\NormalTok{Resultat }\OtherTok{\textless{}{-}}\NormalTok{ a }\SpecialCharTok{+}\NormalTok{ b }\SpecialCharTok{+}\NormalTok{ c }\SpecialCharTok{+}\NormalTok{d }\SpecialCharTok{+}\NormalTok{e}

\NormalTok{Text }\OtherTok{\textless{}{-}} \StringTok{"Hello students, the result is "}
\end{Highlighting}
\end{Shaded}

Rmarkdown permet combiner text et code: Hello students, the result is
78.3333333

\begin{infobox}
\begin{itemize}
\item
  Les noms d'objets peuvent contenir des lettres, des chiffres, les
  symboles \texttt{.} et \texttt{\_}.
\item
  Ils ne peuvent pas commencer par un chiffre.
\item
  R fait la différence entre minuscules et majuscules dans les noms
  d'objets, ce qui signifie que \texttt{x} et \texttt{X} seront deux
  objets différents, tout comme \texttt{resultat} et \texttt{Resultat}.
\end{itemize}

Conseils:

\begin{itemize}
\item
  Il est préférable d'éviter les majuscules (pour les risques d'erreur)
  et les caractères accentués (pour des questions d'encodage) dans les
  noms d'objets.
\item
  De même, il faut essayer de trouver un équilibre entre clarté du nom
  (comprendre à quoi sert l'objet, ce qu'il contient) et sa longueur.
  Par exemple, on préfèrera comme nom d'objet \texttt{taille\_conj1} à
  \texttt{taille\_du\_conjoint\_numero\_1} (trop long) ou à \texttt{t1}
  (pas assez explicite).
\end{itemize}
\end{infobox}

\hypertarget{vecteurs}{%
\subsection{Vecteurs}\label{vecteurs}}

Imaginons maintenant qu'on a demandé la taille en centimètres de 5
personnes et qu'on souhaite calculer leur taille moyenne. On pourrait
créer autant d'objets que de tailles et faire l'opération mathématique
qui va bien :

\begin{Shaded}
\begin{Highlighting}[]
\NormalTok{taille1 }\OtherTok{\textless{}{-}} \DecValTok{156}
\NormalTok{taille2 }\OtherTok{\textless{}{-}} \DecValTok{164}
\NormalTok{taille3 }\OtherTok{\textless{}{-}} \DecValTok{197}
\NormalTok{taille4 }\OtherTok{\textless{}{-}} \DecValTok{147}
\NormalTok{taille5 }\OtherTok{\textless{}{-}} \DecValTok{173}
\NormalTok{(taille1 }\SpecialCharTok{+}\NormalTok{ taille2 }\SpecialCharTok{+}\NormalTok{ taille3 }\SpecialCharTok{+}\NormalTok{ taille4 }\SpecialCharTok{+}\NormalTok{ taille5) }\SpecialCharTok{/} \DecValTok{5}
\end{Highlighting}
\end{Shaded}

\begin{verbatim}
## [1] 167.4
\end{verbatim}

Cette manière de faire n'est évidemment pas pratique du tout. On va
plutôt stocker l'ensemble de nos tailles dans un seul objet, de type
\emph{vecteur}, avec la syntaxe suivante:

\begin{Shaded}
\begin{Highlighting}[]
\NormalTok{tailles }\OtherTok{\textless{}{-}} \FunctionTok{c}\NormalTok{(}\DecValTok{156}\NormalTok{, }\DecValTok{164}\NormalTok{, }\DecValTok{197}\NormalTok{, }\DecValTok{147}\NormalTok{, }\DecValTok{173}\NormalTok{)}
\NormalTok{tailles}
\end{Highlighting}
\end{Shaded}

\begin{verbatim}
## [1] 156 164 197 147 173
\end{verbatim}

\begin{Shaded}
\begin{Highlighting}[]
\CommentTok{\# Taille en metres}
\NormalTok{tailles\_m }\OtherTok{\textless{}{-}}\NormalTok{ tailles }\SpecialCharTok{/} \DecValTok{100}
\NormalTok{tailles\_m}
\end{Highlighting}
\end{Shaded}

\begin{verbatim}
## [1] 1.56 1.64 1.97 1.47 1.73
\end{verbatim}

\begin{Shaded}
\begin{Highlighting}[]
\CommentTok{\#Cela fonctionne pour toutes les opérations de base :}

\NormalTok{tailles }\SpecialCharTok{+} \DecValTok{10}
\end{Highlighting}
\end{Shaded}

\begin{verbatim}
## [1] 166 174 207 157 183
\end{verbatim}

\begin{Shaded}
\begin{Highlighting}[]
\NormalTok{tailles}\SpecialCharTok{\^{}}\DecValTok{2}
\end{Highlighting}
\end{Shaded}

\begin{verbatim}
## [1] 24336 26896 38809 21609 29929
\end{verbatim}

L'avantage d'un vecteur est que lorsqu'on lui applique une opération,
celle-ci s'applique à toutes les valeurs qu'il contient.

Imaginons maintenant qu'on a aussi demandé aux cinq mêmes personnes leur
poids en kilos. On peut alors créer un deuxième vecteur :

\begin{Shaded}
\begin{Highlighting}[]
\NormalTok{poids }\OtherTok{\textless{}{-}} \FunctionTok{c}\NormalTok{(}\DecValTok{45}\NormalTok{, }\DecValTok{59}\NormalTok{, }\DecValTok{110}\NormalTok{, }\DecValTok{44}\NormalTok{, }\DecValTok{88}\NormalTok{)}
\end{Highlighting}
\end{Shaded}

On peut alors effectuer des calculs utilisant nos deux vecteurs
\texttt{tailles} et \texttt{poids}.

On peut par exemple calculer l'indice de masse corporelle (IMC) de
chacun de nos enquêtés en divisant leur poids en kilo par leur taille en
mètre au carré :

\begin{Shaded}
\begin{Highlighting}[]
\CommentTok{\# option 1}
\NormalTok{imc }\OtherTok{\textless{}{-}}\NormalTok{ poids }\SpecialCharTok{/}\NormalTok{ (tailles }\SpecialCharTok{/} \DecValTok{100}\NormalTok{) }\SpecialCharTok{\^{}} \DecValTok{2}
\NormalTok{imc}
\end{Highlighting}
\end{Shaded}

\begin{verbatim}
## [1] 18.49112 21.93635 28.34394 20.36189 29.40292
\end{verbatim}

\begin{Shaded}
\begin{Highlighting}[]
\CommentTok{\# option 2}
\NormalTok{imc }\OtherTok{\textless{}{-}}\NormalTok{ poids }\SpecialCharTok{/}\NormalTok{ (tailles\_m) }\SpecialCharTok{\^{}} \DecValTok{2}
\NormalTok{imc}
\end{Highlighting}
\end{Shaded}

\begin{verbatim}
## [1] 18.49112 21.93635 28.34394 20.36189 29.40292
\end{verbatim}

Un vecteur peut contenir des nombres, mais il peut aussi contenir du
texte.

Imaginons qu'on a demandé aux 5 mêmes personnes leur niveau de diplôme :
on peut regrouper l'information dans un vecteur de \emph{chaînes de
caractères}. Une chaîne de caractère contient du texte libre, délimité
par des guillemets simples ou doubles :

\begin{Shaded}
\begin{Highlighting}[]
\NormalTok{diplome }\OtherTok{\textless{}{-}} \FunctionTok{c}\NormalTok{(}\StringTok{"CAP"}\NormalTok{, }\StringTok{"Bac"}\NormalTok{, }\StringTok{"Bac+2"}\NormalTok{, }\StringTok{"CAP"}\NormalTok{, }\StringTok{"Bac+3"}\NormalTok{)}
\NormalTok{diplome}
\end{Highlighting}
\end{Shaded}

\begin{verbatim}
## [1] "CAP"   "Bac"   "Bac+2" "CAP"   "Bac+3"
\end{verbatim}

\hypertarget{opuxe9rateur}{%
\subsection{\texorpdfstring{Opérateur
\texttt{{[}{]}}}{Opérateur {[}{]}}}\label{opuxe9rateur}}

L'opérateur \texttt{{[}{]}}, permet donc la sélection d'éléments d'un
vecteur. On peut accéder à un élément particulier d'un vecteur en
faisant suivre le nom du vecteur de crochets contenant le numéro de
l'élément désiré. Par exemple :

\begin{Shaded}
\begin{Highlighting}[]
\NormalTok{diplome[}\DecValTok{1}\NormalTok{]}
\end{Highlighting}
\end{Shaded}

\begin{verbatim}
## [1] "CAP"
\end{verbatim}

\begin{Shaded}
\begin{Highlighting}[]
\NormalTok{diplome[}\DecValTok{2}\NormalTok{]}
\end{Highlighting}
\end{Shaded}

\begin{verbatim}
## [1] "Bac"
\end{verbatim}

\begin{Shaded}
\begin{Highlighting}[]
\CommentTok{\#diplome[c(1,2,3)]}
\end{Highlighting}
\end{Shaded}

\hypertarget{lopuxe9rateur}{%
\subsection{\texorpdfstring{L'opérateur
\texttt{:}}{L'opérateur :}}\label{lopuxe9rateur}}

Il permet de générer rapidement un vecteur comprenant tous les nombres
entre deux valeurs, opération assez courante sous R :

\begin{Shaded}
\begin{Highlighting}[]
\NormalTok{diplome[}\DecValTok{1}\SpecialCharTok{:}\DecValTok{3}\NormalTok{]}
\end{Highlighting}
\end{Shaded}

\begin{verbatim}
## [1] "CAP"   "Bac"   "Bac+2"
\end{verbatim}

\begin{Shaded}
\begin{Highlighting}[]
\CommentTok{\# Vecteurs très longes}
\NormalTok{x }\OtherTok{\textless{}{-}} \DecValTok{1}\SpecialCharTok{:}\DecValTok{200}
\NormalTok{x}
\end{Highlighting}
\end{Shaded}

\begin{verbatim}
##   [1]   1   2   3   4   5   6   7   8   9  10  11  12  13  14  15  16  17  18
##  [19]  19  20  21  22  23  24  25  26  27  28  29  30  31  32  33  34  35  36
##  [37]  37  38  39  40  41  42  43  44  45  46  47  48  49  50  51  52  53  54
##  [55]  55  56  57  58  59  60  61  62  63  64  65  66  67  68  69  70  71  72
##  [73]  73  74  75  76  77  78  79  80  81  82  83  84  85  86  87  88  89  90
##  [91]  91  92  93  94  95  96  97  98  99 100 101 102 103 104 105 106 107 108
## [109] 109 110 111 112 113 114 115 116 117 118 119 120 121 122 123 124 125 126
## [127] 127 128 129 130 131 132 133 134 135 136 137 138 139 140 141 142 143 144
## [145] 145 146 147 148 149 150 151 152 153 154 155 156 157 158 159 160 161 162
## [163] 163 164 165 166 167 168 169 170 171 172 173 174 175 176 177 178 179 180
## [181] 181 182 183 184 185 186 187 188 189 190 191 192 193 194 195 196 197 198
## [199] 199 200
\end{verbatim}

Des questions?\ldots{}

\hypertarget{fonctions}{%
\subsection{Fonctions}\label{fonctions}}

\begin{itemize}
\item
  Deuxième concept de base de R.
\item
  On utilise des fonctions pour effectuer des calculs, obtenir des
  résultats et accomplir des actions.
\item
  Une fonction a:

  \begin{enumerate}
  \def\labelenumi{\arabic{enumi}.}
  \tightlist
  \item
    un \emph{nom},
  \item
    Elle prend en entrée entre parenthèses un ou plusieurs
    \emph{arguments} (ou \emph{paramètres}), 3.Retourne un
    \emph{résultat}.
  \end{enumerate}
\end{itemize}

Exemple: Si on veut connaître le nombre d'éléments du vecteur
\texttt{tailles} que nous avons construit précédemment, on peut utiliser
la fonction \texttt{length}, de cette manière :

\begin{Shaded}
\begin{Highlighting}[]
\FunctionTok{length}\NormalTok{(tailles)}
\end{Highlighting}
\end{Shaded}

\begin{verbatim}
## [1] 5
\end{verbatim}

\begin{Shaded}
\begin{Highlighting}[]
\FunctionTok{min}\NormalTok{(tailles)}
\end{Highlighting}
\end{Shaded}

\begin{verbatim}
## [1] 147
\end{verbatim}

\begin{Shaded}
\begin{Highlighting}[]
\FunctionTok{max}\NormalTok{(tailles)}
\end{Highlighting}
\end{Shaded}

\begin{verbatim}
## [1] 197
\end{verbatim}

\begin{Shaded}
\begin{Highlighting}[]
\FunctionTok{sum}\NormalTok{(tailles)}
\end{Highlighting}
\end{Shaded}

\begin{verbatim}
## [1] 837
\end{verbatim}

\begin{Shaded}
\begin{Highlighting}[]
\FunctionTok{mean}\NormalTok{(tailles)}
\end{Highlighting}
\end{Shaded}

\begin{verbatim}
## [1] 167.4
\end{verbatim}

\begin{Shaded}
\begin{Highlighting}[]
\FunctionTok{range}\NormalTok{(tailles)}
\end{Highlighting}
\end{Shaded}

\begin{verbatim}
## [1] 147 197
\end{verbatim}

\begin{Shaded}
\begin{Highlighting}[]
\CommentTok{\# Voir l\textquotesingle{}aide des fonctions}
\NormalTok{?length}
\NormalTok{?max}
\FunctionTok{help}\NormalTok{(}\StringTok{"mean"}\NormalTok{)}
\end{Highlighting}
\end{Shaded}

\begin{itemize}
\item
  \texttt{length}: est le nom de la fonction, on l'appelle en lui
  passant un argument entre parenthèses (en l'occurrence notre vecteur
  \texttt{tailles}), et elle nous renvoie un résultat, à savoir le
  nombre d'éléments du vecteur passé en paramètre.
\item
  \texttt{min} et \texttt{max} retournent respectivement les valeurs
  minimales et maximales d'un vecteur de nombres.
\item
  \texttt{mean}: calcule et retourne la moyenne d'un vecteur de nombres
  :
\item
  \texttt{sum} retourne la somme de tous les éléments du vecteur :
\item
  \texttt{range} (étendue) renvoie un vecteur de deux nombres, le
  minimum et le maximum :
\end{itemize}

\begin{Shaded}
\begin{Highlighting}[]
\NormalTok{diplome }\OtherTok{\textless{}{-}} \FunctionTok{c}\NormalTok{(}\StringTok{"CAP"}\NormalTok{, }\StringTok{"Bac"}\NormalTok{, }\StringTok{"Bac+2"}\NormalTok{, }\StringTok{"CAP"}\NormalTok{, }\StringTok{"Bac+3"}\NormalTok{)}
\FunctionTok{unique}\NormalTok{(diplome)}
\end{Highlighting}
\end{Shaded}

\begin{verbatim}
## [1] "CAP"   "Bac"   "Bac+2" "Bac+3"
\end{verbatim}

\begin{Shaded}
\begin{Highlighting}[]
\FunctionTok{nchar}\NormalTok{(diplome)}
\end{Highlighting}
\end{Shaded}

\begin{verbatim}
## [1] 3 3 5 3 5
\end{verbatim}

\begin{itemize}
\tightlist
\item
  \texttt{unique}, qui supprime toutes les valeurs en double dans un
  vecteur, qu'il s'agisse de nombres ou de chaînes de caractères :
\end{itemize}

\hypertarget{arguments}{%
\subsubsection{Arguments}\label{arguments}}

\begin{itemize}
\item
  Une fonction peut prendre plusieurs arguments, dans ce cas on les
  indique toujours entre parenthèses, séparés par des virgules.
\item
  Un exemple de fonction acceptant plusieurs arguments : la fonction
  \texttt{c}, qui combine l'ensemble de ses arguments en un
  vecteur\footnote{\texttt{c} est l'abbréviation de \emph{combine}, son
    nom est très court car on l'utilise très souvent} :
\end{itemize}

\begin{Shaded}
\begin{Highlighting}[]
\NormalTok{tailles }\OtherTok{\textless{}{-}} \FunctionTok{c}\NormalTok{(}\DecValTok{156}\NormalTok{, }\DecValTok{164}\NormalTok{, }\DecValTok{197}\NormalTok{, }\DecValTok{181}\NormalTok{, }\DecValTok{173}\NormalTok{)}
\end{Highlighting}
\end{Shaded}

Ici, \texttt{c} est appelée en lui passant cinq arguments, les cinq
tailles séparées par des virgules, et elle renvoie un vecteur numérique
regroupant ces cinq valeurs.

Supposons maintenant que dans notre vecteur \texttt{tailles} nous avons
une valeur manquante (une personne a refusé de répondre, ou notre mètre
mesureur était en panne). On symbolise celle-ci dans R avec le code
interne \texttt{NA} :

\begin{Shaded}
\begin{Highlighting}[]
\NormalTok{tailles }\OtherTok{\textless{}{-}} \FunctionTok{c}\NormalTok{(}\DecValTok{156}\NormalTok{, }\DecValTok{164}\NormalTok{, }\DecValTok{197}\NormalTok{, }\ConstantTok{NA}\NormalTok{, }\DecValTok{173}\NormalTok{)}
\NormalTok{tailles}
\end{Highlighting}
\end{Shaded}

\begin{verbatim}
## [1] 156 164 197  NA 173
\end{verbatim}

\begin{verbatim}
\texttt{NA} est l'abbréviation de \emph{Not available}, non disponible.
Cette valeur particulière peut être utilisée pour indiquer une valeur
manquante, qu'il s'agisse d'un nombre, d'une chaîne de caractères, etc.

\end{verbatim}

\begin{infobox}
t
\end{infobox}

Si je calcule maintenant la taille moyenne à l'aide de la fonction
\texttt{mean}, j'obtiens :

\begin{Shaded}
\begin{Highlighting}[]
\CommentTok{\# Il considère alors que cette moyenne est elle{-}même "non disponible" et renvoie donc comme résultat \textasciigrave{}NA\textasciigrave{}.}
\FunctionTok{mean}\NormalTok{(tailles)}
\end{Highlighting}
\end{Shaded}

\begin{verbatim}
## [1] NA
\end{verbatim}

\begin{Shaded}
\begin{Highlighting}[]
\CommentTok{\# Ceci se fait en ajoutant un argument supplémentaire, nommé \textasciigrave{}na.rm\textasciigrave{} (abbréviation de *NA remove*, "enlever les NA"), et de lui attribuer la valeur \textasciigrave{}TRUE\textasciigrave{} (code interne de R signifiant *vrai*) :}

\FunctionTok{mean}\NormalTok{(tailles, }\AttributeTok{na.rm =} \ConstantTok{TRUE}\NormalTok{)}
\end{Highlighting}
\end{Shaded}

\begin{verbatim}
## [1] 172.5
\end{verbatim}

\begin{rmdnote}
Lorsqu'on passe un argument à une fonction de cette manière,
c'est-à-dire sous la forme \texttt{nom\ =\ valeur}, on parle
d'\emph{argument nommé}.
\end{rmdnote}

\hypertarget{packages}{%
\subsection{Packages}\label{packages}}

\begin{itemize}
\item
  R étant un logiciel libre, il bénéficie d'un développement
  communautaire riche et dynamique.
\item
  L'installation de base de R permet de faire énormément de choses, mais
  le langage dispose en plus d'un système d'extensions permettant
  d'ajouter facilement de nouvelles fonctionnalités.
\item
  La plupart des extensions sont développées et maintenues par la
  communauté des utilisateurs de R, et diffusées via un réseau de
  serveurs nommé CRAN (Comprehensive R Archive Network).
\end{itemize}

\begin{Shaded}
\begin{Highlighting}[]
\CommentTok{\# Options 1}
\FunctionTok{install.packages}\NormalTok{(}\StringTok{"questionr"}\NormalTok{)}
\FunctionTok{install.packages}\NormalTok{(}\StringTok{"tidyverse"}\NormalTok{)}

\CommentTok{\# Option 2 {-}{-}\textgreater{} Paneau}
\end{Highlighting}
\end{Shaded}

il faut la ``charger'' avant de pouvoir utiliser les fonctions qu'elle
propose. Ceci se fait avec la fonction \texttt{library}.

\begin{Shaded}
\begin{Highlighting}[]
\FunctionTok{library}\NormalTok{(questionr)}
\FunctionTok{library}\NormalTok{(tidyverse)}
\end{Highlighting}
\end{Shaded}

\begin{verbatim}
## -- Attaching packages --------------------------------------- tidyverse 1.3.0 --
\end{verbatim}

\begin{verbatim}
## v ggplot2 3.3.3     v purrr   0.3.4
## v tibble  3.0.6     v dplyr   1.0.4
## v tidyr   1.1.2     v stringr 1.4.0
## v readr   1.4.0     v forcats 0.5.1
\end{verbatim}

\begin{verbatim}
## -- Conflicts ------------------------------------------ tidyverse_conflicts() --
## x dplyr::filter() masks stats::filter()
## x dplyr::lag()    masks stats::lag()
\end{verbatim}

Prenons une Pause de 10 min !!!

\end{document}
